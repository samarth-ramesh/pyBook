\chapter{While Loops}
In this chapter, we shall discuss the bread and butter of control structure, the \emph{while} loop. 
\section{While Loops}
While loops are the bread and butter of infinite loops, most commonly seen in games and the like, where valid user input is needed.
\subsection{Syntax}
Like the for loop, and the conditional, the while loop too creates a new code block.
The syntax for a while loop is as follows:
\begin{quote}
while (condition): \newline
\tab \textlangle{}your code here\textrangle{}
\end{quote}
Note the use of \emph{:} and an additional indentation.

\subsection{Usage}
To calculate the mean of the first 100 numbers, we can use
\begin{quote}
x = 1 \newline
sum = 0 \newline
while x $<$ 101: $\#$note the use of 101, rather than $<$= 100 \newline
\tab sum = sum + x \newline
\tab x = x + 1 \newline
print(sum/100)
\end{quote}

\subsection{Explanation}
The while loop simply put executes the code within it's code block until such a time that the condition evaluates to False. (Note, if the condition evaluates to false at the starting, the code within never gets executed. To ensure that the code executes atleast once, use \emph{do…while})\newline
Below, you can see a pictographic representation of the while loop\newline \newline
 % =================================================
% Set up a few colours
\colorlet{lcfree}{Green3}
\colorlet{lcnorm}{Blue3}
\colorlet{lccong}{Red3}
% -------------------------------------------------
% Set up a new layer for the debugging marks, and make sure it is on
% top
\pgfdeclarelayer{marx}
\pgfsetlayers{main,marx}
% A macro for marking coordinates (specific to the coordinate naming
% scheme used here). Swap the following 2 definitions to deactivate
% marks.
\providecommand{\cmark}[2][]{%
  \begin{pgfonlayer}{marx}
    \node [nmark] at (c#2#1) {#2};
  \end{pgfonlayer}{marx}
  } 
\providecommand{\cmark}[2][]{\relax} 
% -------------------------------------------------
\begin{tikzpicture}[%
    >=triangle 60,              % Nice arrows; your taste may be different
    start chain=going below,    % General flow is top-to-bottom
    node distance=6mm and 60mm, % Global setup of box spacing
    every join/.style={norm},   % Default linetype for connecting boxes
    ]
\tikzset{
  base/.style={draw, on chain, on grid, align=center, minimum height=4ex},
  proc/.style={base, rectangle, text width=8em},
  test/.style={base, diamond, aspect=2, text width=5em},
  term/.style={proc, rounded corners},
  % coord node style is used for placing corners of connecting lines
  coord/.style={coordinate, on chain, on grid, node distance=6mm and 25mm},
  % nmark node style is used for coordinate debugging marks
  nmark/.style={draw, cyan, circle, font={\sffamily\bfseries}},
  % -------------------------------------------------
  % Connector line styles for different parts of the diagram
  norm/.style={->, draw, lcnorm},
  free/.style={->, draw, lcfree},
  cong/.style={->, draw, lccong},
  it/.style={font={\small\itshape}}
}
\node[term, it] (start){While Loop Begins};
\node[test, join] (t1) {Is condition true?};
\node[proc] (p1) {code in block};


\node [coord] (c1)  {}; 
\node[coord, right of = t1] (c2) {};

\node[term, below of= c1, it] (e1) {End Of While Loop};

\path (t1.south) to node [anchor=east]{yes} (p1);
\draw[->, lcnorm] (t1.south) -- (p1);
\draw[->,lcnorm] (c1.south) -| (t1.west);
\draw[-, lcnorm] (p1.south) -- (c1);

\path (t1.west) to node [anchor=west, xshift = 11em]{no} (e1);
\draw[->, lccong] (t1.east) |- (e1);
\end{tikzpicture}
 \newline

\subsection{Examples}
\begin{enumerate}
\item Convert all integral values of celsius into fahrenhiet between 0 and 100 both included.


We can solve this problem in many different ways. Since we are discussing \emph{while} loops, we will use it to solve our problem.
\begin{quote}
x = 0 \newline
FtoCFactor = 1/1.8 \newline
while (x $<$ 101):\newline
\tab print('The Fahranhiet equivalent of', x, 'is', ((x-32)/FtoCFactor))\newline
\tab x = x+1 \# Note, we can also use x += 1 \newline
\end{quote}
A note about the 5\textsuperscript{th} line, We used $x = x + 1$ because Without it, the while loop will never terminate (because x will never increase)

\item Print out all the possible IPv4 addresses of the form $192.168.0.x$ and $192.168.1.y$

IPv4 addresses are basically 4 numbers between 0 and 255 (both included) seperated by dots.
Some famous IPv4 addresses include 127.0.0.1 (the loopback address), 8.8.8.8 (google public dns) and 192.168.0.1 (a standard home network router IP address)

We can as always solve this problem in multiple different ways. Since we are asked to find \emph{two} subnets that are very close to each other, we can use a while loop for the inner prefix. Similiarly, we can use a \emph{nested while loop} for the outer prefix.

\begin{quote}
net = '192.168.' \newline
counter1 = 0 \newline
counter2 = 0 \newline
while (counter1 $<$ 2): \newline
\tab while (counter2 $<$ 256): \newline
\tab \tab print(net + counter1 + '.' + counter2) \newline
\tab \tab counter2 += 1 \newline 
\tab counter1 += 1 \newline
\end{quote}
And there you have a nested while loop, the bread and butter of many algorithms.
\end{enumerate}
\newpage
\subsection{Excercises}

\begin{enumerate}
\item Write a script to calculate the n\textsuperscript{th} power of 8 (for the purposes of testing, use $n=9$), using a while loop
\item Write a script to calculate the factorial of 15. (n factorial = $1\times 2\times 3...(n-1)\times (n)$)
\end{enumerate}
