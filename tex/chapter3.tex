\chapter{More Controlling}
So far, we have completed the bread and butter of basic programs, the if's, the else's and the for's…\newline
Now, we shall take on some more useful peices of control flow, namely \emph{while}, \emph{break}, \emph{continue}
\section{While Loops}
While loops are the bread and butter of infinite loops, most commonly seen in games and the like, where valid user input is needed.
\subsection{Syntax}
Like the for loop, and the conditional, the while loop too creates a new code block.
The syntax for a while loop is as follows:
\begin{quote}
while (condition): \newline
\tab \textlangle{}your code here\textrangle{}
\end{quote}
Note the use of \emph{:} and an additional indentation.

\subsection{Usage}
To calculate the mean of the first 100 numbers, we can use
\begin{quote}
x = 1 \newline
sum = 0 \newline
while x $<$ 101: $\#$note the use of 101, rather than $<$= 100 \newline
\tab sum = sum + x \newline
\tab x = x + 1 \newline
print(sum/100)
\end{quote}

Now, Try to rewrite the Fahrenhiet to Celsius convertor using a while loop
